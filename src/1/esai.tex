\documentclass[11pt, a4paper]{article}

% Konfigurasi Bahasa dan Encoding
\usepackage[utf8]{inputenc}
\usepackage[indonesian]{babel} % Mengatur pemenggalan kata bahasa Indonesia

% Pengaturan Halaman dan Margin (Standar 1 inci)
\usepackage{geometry}
\geometry{margin=1in}

% Paket untuk memberi jarak antar paragraf (agar terlihat terpisah)
\usepackage{parskip} 

% Paket tambahan untuk formatting (opsional)
\usepackage{xcolor}
\usepackage{hyperref}

\begin{document}

\title{Motivasi Bergabung dengan Aksantara}
\author{Simarmata Markus Nicholas Santos - 16025088}
\date{Dirilis: 1 Februari 2026}
\maketitle

Semuanya berawal dari video YouTube yang saya tonton sekitar satu tahun lalu. Video tersebut adalah video dari Red Bull yang berjudul ``World's Fastest Camera Drone Vs F1 Car (ft. Max Verstappen)''. Melihat visual yang bisa didapatkan dari drone membangun suatu ketertarikan dalam diri saya untuk mengulik lebih lanjut. Selain itu juga, dalam beberapa buku yang saya baca dan film yang saya tonton, masa depan dunia militer adalah persenjataan nirawak seperti drone. Dan sejak itulah saya memutuskan arah karier saya (saat itu saya berada di bangku SMA kelas 12). Saya mempertimbangkan beberapa jurusan kuliah yang saya ingin pilih, tetapi yang saya pilih adalah Teknik Elektro. Mengapa? Karena pertama, saya suka mengulik elektronik waktu kecil (lumayan \textit{tech enthusiast}); kedua, Teknik Elektro masih ada \textit{programming}-nya (saya juga suka \textit{programming} meskipun tidak handal); dan terakhir, saya tidak tertarik dalam dunia penerbangan secara keseluruhan. Akhirnya saya memfokuskan diri untuk belajar UTBK agar masuk ke STEI-R ITB.

Sayangnya, takdir berkata lain, saya ditolak dari STEI-R ITB dan masuk ke pilihan dua. Saya sudah merencanakan dari awal bahwa \textit{career path}-nya akan cukup berbeda jika saya diterima pilihan ke dua, yakni FMIPA. Saya tidak ingin memaksakan mempelajari dunia elektronik dan penerbangan secara mandiri karena dunia perkuliahan saja sudah berat, sehingga saya memutuskan untuk mengubur mimpi saya untuk bergerak di industri UAV dan fokus ke arah \textit{scientist} (saya ingin mengikuti jalannya Elon Musk dari fisika menjadi \textit{businessman}). Namun, setelah berkuliah di ITB selama beberapa minggu, saya diberi tahu oleh teman saya yang dari FTMD bahwa dia tertarik pada suatu UKM teknis yakni Aksantara. Saya tertarik saat mendengar cerita teman saya tentang Aksantara karena ini terasa seperti ada peluang untuk mendapatkan \textit{second chance}-ku. Saat aku mendalami sedikit tentang Aksantara, aku jadi tertarik untuk masuk dan merasa, ``jika aku keterima, aku menjadi punya alasan lagi untuk mendalami ketertarikanku''. Akhirnya, saya memutuskan untuk mendaftar Aksantara jika pendaftaran sudah buka.

Saya membaca seluruh departemen yang ada di Aksantara, dan setelah mempertimbangkan, saya memilih divisi RSC. Alasan saya memilih divisi RSC adalah karena selama berjalannya waktu perkuliahan, saya sempat mempelajari tentang \textit{World Models}, yakni AI yang bisa mensimulasikan cara kerja dunia (bukan \textit{generative AI}), dan topik ini berhubungan dengan jurusan yang aku mau. Dari situlah aku mulai mendalami Python dan \textit{Data Science}. Sehingga, karena saya sedang mendalami Python dan \textit{Data Science}, ditambah dulu memiliki ketertarikan \textit{programming}, saya memilih divisi RSC. Plus, agar memiliki \textit{skill technical} juga, bukan \textit{scientist} saja.

Terakhir, jika saya keterima di Aksantara, pastinya pertama saya akan berkontribusi sebaik mungkin di tim di mana saya ditempatkan. Selain itu, karena di Aksantara (terutama divisi \textit{technical}) kebanyakan dari jurusan teknik, mungkin saya bisa memberikan inovasi dari dunia sains yang mungkin hanya sekadar teori, tetapi bisa saja diwujudkan. Saya juga ingin melanjutkan pendalaman saya di \textit{Data Science} untuk menganalisis drone sehingga bisa memberikan solusi atas permasalahan-permasalahan, sehingga mengembangkan drone menjadi lebih baik lagi. Jadi, saya bisa melakukan simbiosis mutualisme dengan Aksantara.

\end{document}